\documentclass[]{book}
\usepackage{lmodern}
\usepackage{amssymb,amsmath}
\usepackage{ifxetex,ifluatex}
\usepackage{fixltx2e} % provides \textsubscript
\ifnum 0\ifxetex 1\fi\ifluatex 1\fi=0 % if pdftex
  \usepackage[T1]{fontenc}
  \usepackage[utf8]{inputenc}
\else % if luatex or xelatex
  \ifxetex
    \usepackage{mathspec}
  \else
    \usepackage{fontspec}
  \fi
  \defaultfontfeatures{Ligatures=TeX,Scale=MatchLowercase}
\fi
% use upquote if available, for straight quotes in verbatim environments
\IfFileExists{upquote.sty}{\usepackage{upquote}}{}
% use microtype if available
\IfFileExists{microtype.sty}{%
\usepackage{microtype}
\UseMicrotypeSet[protrusion]{basicmath} % disable protrusion for tt fonts
}{}
\usepackage[margin=1in]{geometry}
\usepackage{hyperref}
\hypersetup{unicode=true,
            pdftitle={R과 함께하는 정보 과학},
            pdfauthor={이광춘 - R Meetup 운영},
            pdfborder={0 0 0},
            breaklinks=true}
\urlstyle{same}  % don't use monospace font for urls
\usepackage{natbib}
\bibliographystyle{apalike}
\usepackage{longtable,booktabs}
\usepackage{graphicx,grffile}
\makeatletter
\def\maxwidth{\ifdim\Gin@nat@width>\linewidth\linewidth\else\Gin@nat@width\fi}
\def\maxheight{\ifdim\Gin@nat@height>\textheight\textheight\else\Gin@nat@height\fi}
\makeatother
% Scale images if necessary, so that they will not overflow the page
% margins by default, and it is still possible to overwrite the defaults
% using explicit options in \includegraphics[width, height, ...]{}
\setkeys{Gin}{width=\maxwidth,height=\maxheight,keepaspectratio}
\IfFileExists{parskip.sty}{%
\usepackage{parskip}
}{% else
\setlength{\parindent}{0pt}
\setlength{\parskip}{6pt plus 2pt minus 1pt}
}
\setlength{\emergencystretch}{3em}  % prevent overfull lines
\providecommand{\tightlist}{%
  \setlength{\itemsep}{0pt}\setlength{\parskip}{0pt}}
\setcounter{secnumdepth}{5}
% Redefines (sub)paragraphs to behave more like sections
\ifx\paragraph\undefined\else
\let\oldparagraph\paragraph
\renewcommand{\paragraph}[1]{\oldparagraph{#1}\mbox{}}
\fi
\ifx\subparagraph\undefined\else
\let\oldsubparagraph\subparagraph
\renewcommand{\subparagraph}[1]{\oldsubparagraph{#1}\mbox{}}
\fi

%%% Use protect on footnotes to avoid problems with footnotes in titles
\let\rmarkdownfootnote\footnote%
\def\footnote{\protect\rmarkdownfootnote}

%%% Change title format to be more compact
\usepackage{titling}

% Create subtitle command for use in maketitle
\newcommand{\subtitle}[1]{
  \posttitle{
    \begin{center}\large#1\end{center}
    }
}

\setlength{\droptitle}{-2em}
  \title{R과 함께하는 정보 과학}
  \pretitle{\vspace{\droptitle}\centering\huge}
  \posttitle{\par}
  \author{이광춘 - R Meetup 운영}
  \preauthor{\centering\large\emph}
  \postauthor{\par}
  \predate{\centering\large\emph}
  \postdate{\par}
  \date{2018-02-03}

\usepackage{booktabs}

\begin{document}
\maketitle

{
\setcounter{tocdepth}{1}
\tableofcontents
}
\chapter*{\texorpdfstring{기계와의 경쟁 \footnote{\href{http://news.mk.co.kr/newsRead.php?year=2018\&no=58432}{`사피엔스'
  저자 유발 하라리 ``인간을 해킹하는 시대가 온다'', ``머신러닝·AI·생물학
  발전\ldots{}뇌과학 이해도 한층 높여''}}}{기계와의 경쟁 }}\label{--joongang-yuval}
\addcontentsline{toc}{chapter}{기계와의 경쟁 }

2014년부터 \textbf{xwMOOC} 프로젝트를 진행하면서
\href{https://software-carpentry.org/}{소프트웨어 카펜트리(Software
Carpentry)} 와 \href{https://datacarpentry.org/}{데이터 카펜트리(Data
Carpentry)} 비영리 공공 프로젝트가 많은 영감을 주었다. 그러던 와중에
2018년부터 \textbf{소프트웨어 교육} 이 초중등 교육과정에 의무화되었고,
알파고가 이세돌을 이긴 \textbf{알파고 쇼크} 가 한국사회에 엄청난 파장을
일으켰다. 그와 더불어 청년, 중장년, 노년 할 것 없이 실업률이 사회적
문제로 대두되면서 기계가 인간의 직업을 빼앗아가는 주범으로 주목되고 있는
한편, \textbf{인공지능} 기술을 밑에 깔고 있는 다양한 제품과 서비스가
쏟아지면서 우리의 삶을 그 어느 때보다 풍요롭게 만들고 있다.

컴퓨팅 사고력, 데이터 과학, 인공지능, 로봇/기계를 이해하는 사람과 그렇지
못한 사람간에 삶의 질 차이는 점점더 현격히 벌어질 것이다. 지금이라도
늦지 않았다. 늦었다는 것을 알아차렸을 때가 가장 빠른 시점이다.

유발 하라리 교수가 지적했듯이 데이터가 권력과 부의 원천이 되는 세상으로
접어들었는데 이에 대해서 컴퓨터와 적절하게 의사소통할 수 있는 언어가
필요하다.
\href{http://statkclee.github.io/data-science/ds-r-lang.html}{\textbf{R}}는
일반인에게 많이 알려져 있지 않지만, 파이썬과 더불어 데이터 프로그래밍에
있어 유구한 역사와 탄탄한 사용자 기반을 가지고 최근들어 혁신적인 변화를
이끌고 있는 언어 중의 하나다.

\begin{quote}
``The future is here, it's just not evenly distributed yet.''\\
\hspace*{0.333em}\hspace*{0.333em}\hspace*{0.333em}\hspace*{0.333em}\hspace*{0.333em}\hspace*{0.333em}\hspace*{0.333em}\hspace*{0.333em}\hspace*{0.333em}\hspace*{0.333em}\hspace*{0.333em}\hspace*{0.333em}\hspace*{0.333em}\hspace*{0.333em}\hspace*{0.333em}\hspace*{0.333em}\hspace*{0.333em}\hspace*{0.333em}\hspace*{0.333em}\hspace*{0.333em}
- William Gibson

``고대에는 '땅'이 가장 중요했고 땅이 소수에게 집중되자 인간은 귀족과
평민으로 구분됐으며, 근대에는 '기계'가 중요해지면서 기계가 소수에게
집중되자 인간은 자본가와 노동자 계급으로 구분됐다''. 이제는
\textbf{데이터}가 또 한번 인류를 구분하는 기준이 될 것이다. 향후
데이터가 소수에게 집중되면 단순 계급에 그치는 게 아니라 데이터를 가진
종과 그렇지 못한 종으로 분류될 것이이다. \footnote{\href{http://news.mk.co.kr/newsRead.php?year=2018\&no=58432}{`사피엔스'
  저자 유발 하라리 ``인간을 해킹하는 시대가 온다'', ``머신러닝·AI·생물학
  발전\ldots{}뇌과학 이해도 한층 높여''}}

~~~~~~~~~~~~~~~~~~~~ - 유발 하라리(Yuval Noah Harari)
\end{quote}

\bibliography{book.bib,packages.bib}


\end{document}
